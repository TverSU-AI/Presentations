\documentclass[intlimits, 12pt, unicode]{beamer}

\usepackage[T2A]{fontenc}
\usepackage[utf8]{inputenc}
\usepackage[russian]{babel}

\usepackage{graphicx}

\usepackage{amssymb}
\usepackage{amsthm}
\usepackage{diagbox}
\usepackage{beamerthemesplit}
\usepackage{xcolor,colortbl}

\usetheme{Warsaw}

\makeatletter
\setbeamertemplate{headline}{%
    \leavevmode%
    \begin{beamercolorbox}[wd=.5\paperwidth,ht=2.5ex,dp=1.5ex]{section in head/foot}%
        \hfill\strut\insertsectionhead\hspace{.5em}\mbox{}%
    \end{beamercolorbox}%
    \begin{beamercolorbox}[wd=.5\paperwidth,ht=2.5ex,dp=1.5ex]{subsection in head/foot}%
        \mbox{}\hspace{.5em}\strut\insertsubsectionhead\hfill%
    \end{beamercolorbox}%
}
\makeatother

\usefonttheme[onlymath]{serif}

\setbeamerfont{institute}{size=\normalsize}

\setbeamercolor{bluetext_color}{fg=blue}
\newcommand{\bluetext}[1]{{\usebeamercolor[fg]{bluetext_color}#1}}

\setbeamertemplate{navigation symbols}{}
\setbeamertemplate{caption}[numbered]

\graphicspath{{figures/}}
\DeclareGraphicsExtensions{.pdf,.png,.jpg}


\setbeamercovered{transparent}

\title[Система компьютерного зрения \dots \nobreak\hfill \insertframenumber\,/\,\inserttotalframenumber]{Система компьютерного зрения для детектирования опасных ситуаций на производстве}
\author{Фомин~Никита~Алексеевич,~М26 группа}
\institute{Научный руководитель: к.ф.-м.н.~Солдатенко~Илья~Сергеевич \\
    Тверской~государственный~университет \\
	Факультет~прикладной~математики~и~кибернетики \\
	Кафедра~информатики~и~информационных~технологий \\
}
\date{
	Тверь ---
	2024
}

\begin{document}
	\maketitle

	\begin{frame}
		\frametitle{Содержание}

		\tableofcontents
	\end{frame}

	\section{Постановка задачи}

\begin{frame}
    \frametitle{Постановка задачи}
    \begin{figure}
        \centering
        \includegraphics[width=1.0\textwidth,keepaspectratio]{problem_formulation_1}
    \end{figure}
\end{frame}

\begin{frame}
    \frametitle{Постановка задачи}
    Начальные предположения:
    \begin{itemize}
        \item Опасная зона для конкретной камеры задана заранее
        \item Детектируем только случай нахождения в опасной зоне без СИЗ
    \end{itemize}
\end{frame}

\begin{frame}
    \frametitle{Архитектура решения}
    \begin{figure}
        \centering
        \includegraphics[width=1.1\textwidth,keepaspectratio]{general_architecture}
    \end{figure}
\end{frame}


	\section{Распознавание и скелетизация людей}
\begin{frame}
    \frametitle{Распознавание людей и их позиций}
    \begin{figure}
        \centering
        \includegraphics[width=0.9\textwidth,keepaspectratio]{skeletization}
    \end{figure}
\end{frame}

%\begin{frame}
%    \frametitle{Модель AlphaPose}
%    \begin{figure}
%        \centering
%        \includegraphics[width=1.0\textwidth,keepaspectratio]{alphapose_1}
%    \end{figure}
%\end{frame}

\begin{frame}
    \frametitle{Модель AlphaPose}
    \begin{figure}
        \centering
        \includegraphics[width=1.0\textwidth,keepaspectratio]{alphapose_2}
    \end{figure}
\end{frame}

\begin{frame}
    \frametitle{Выходной формат данных скелетной модели}
    По заданному кадру алгоритм строит множество $\mathbf{S}$ скелетных моделей.

    Каждый элемент $\mathbf{S}$ имеет вид:
    $$ \left(ID, K, BB\right), $$
    где:
    \begin{itemize}
        \item $ID$ -- уникальный идентификатор человека в кадре
        \item $K$ -- список ключевых точек скелетной модели
        \item $BB$ -- ограничивающая рамка для изображения человека
    \end{itemize}
\end{frame}

\begin{frame}
    \frametitle{Формат представления ключевых точек}
    $$ K = [(x_i, y_i, c_i)]_{i=0}^{N-1}, $$
    где
    \begin{itemize}
        \item $(x_i, y_i)$ -- координаты ключевой точки на изображении
        \item $c_i$ -- уверенность алгоритма, что ключевая точка предсказана корректно
        \item $N$ -- общее число ключевых точек (26~тело, 21~левая~рука, 21~правая~рука)
    \end{itemize}
\end{frame}

\begin{frame}
    \frametitle{Нумерация ключевых точек}
    \begin{figure}
        \begin{minipage}[!h]{0.49\linewidth}
            \centering
            \includegraphics[width=0.7\textwidth,keepaspectratio]{keypoints_map_1}
        \end{minipage}
        \hfill
        \begin{minipage}[!h]{0.49\linewidth}
            \centering
            \includegraphics[width=0.7\textwidth,keepaspectratio]{keypoints_map_2}
        \end{minipage}
    \end{figure}
\end{frame}

	\section{Детектирование нахождения в опасной зоне}
\begin{frame}
    \frametitle{Простейшее определение опасной зоны}
    В простейшем случае, опасную зону можно определить следующим образом.

    Зафиксируем выпуклый многоугольник $A$ на изображении.
    $$ x_{min} = \min\{ x : \exists y (x, y) \in A\} $$
    $$ x_{max} = \max\{ x : \exists y (x, y) \in A\} $$

    Тогда опасной зоной $DZ$ будем называть выпуклую оболочку множества
    $$ A \cup \{(x_{min}, 0), (x_{max}, 0)\} $$
\end{frame}

\begin{frame}
    \frametitle{Определение принадлежности опасной зоне}
    Задан минимальный порог срабатывания $m$ для принадлежности ключевой точки зоне.

    По определению $I_i = 1$, если ключевая точка человека $\left(x_i, y_i, c_i\right)$ находится в опасной зоне.
    $$ I_i = 1 \iff \left(c_i \ge m\right) \wedge \left(\left(x_i, y_i\right) \in DZ\right) $$

    Задана минимальная доля $t$ ключевых точек, принадлежащих опасной зоне, при которой считаем человека находящимся в ней.

    Считаем, что человек находится в опасной зоне, если:
    $$ \frac{\sum I_i}{N} \ge t $$
\end{frame}

\begin{frame}
    \frametitle{Усложнение определения опасной зоны}
    Будем считать, что область пространства, обозреваемого камерой, локально представляет из себя $\mathbb{R}^3$.

    Зафиксируем правильный многоугольник $A$ в плоскости $z = 0$.
    Тогда опасной зоной будем называть множество:
    $$DZ = \{(x, y, z) : (x, y, 0) \in A\} $$
\end{frame}

\begin{frame}
    \frametitle{Детектирование нахождения в опасной зоне}
    \begin{figure}
        \centering
        \includegraphics[width=1.0\textwidth,keepaspectratio]{danger_zone}
    \end{figure}
\end{frame}

\begin{frame}
    \frametitle{Усложнение определения опасной зоны}
    Перейдём от координат в $\mathbb{R}^3$ к обобщённым координатам в $\mathbb{P}^3$. Рассмотрим изображение как проективную плоскость $\mathbb{P}^2$ и найдём отображение $P_{3\times4}$:
    \begin{equation*}
        \left(
        \begin{array}{c}
            x \\
            y \\
            w
        \end{array}
        \right) =
        P_{3\times4}
        \left(
        \begin{array}{c}
            X \\
            Y \\
            Z \\
            T
        \end{array}
        \right)
    \end{equation*}

    Имея матрицу камеры $P_{3\times4}$ можно проверять что ключевые точки скелетной модели находятся внутри призмы представляющей собой опасную зону.
\end{frame}

\begin{frame}
    \frametitle{Пример центральной проекции}
    \begin{figure}
        \centering
        \includegraphics[width=1.0\textwidth,keepaspectratio]{central_projection}
    \end{figure}
\end{frame}

    \section{Принятие решения}
\begin{frame}
    \frametitle{Принятие решения}
    Пусть

    $\left[P_i\right]_{i=0}^{M-1}$ -- результат работы модуля распознавания СИЗов,

    $\left[D_i\right]_{i=0}^{M-1}$ -- результат работы модуля детектирования нахождения в опасной зоне:
    $$ P_i = 1 \iff  \text{i-й человек находится в кадре без СИЗ}$$
    $$ D_i = 1 \iff \text{i-й человек находится в кадре в опасной зоне} $$

    В данном случае $M$ -- число обнаруженных людей в кадре.

    Тогда будем считать ситуацию на текущем кадре опасной, если истинно следующее значение:
    $$ F = \bigvee\limits_{i=0}^{M-1} \left(P_i \wedge D_i\right) $$
\end{frame}

\begin{frame}
    \frametitle{Матрица возможных ответов}
    Для расширения спектра возможных ответов требуется введение индикатора присутствия человека без СИЗ в кадре:
    $$ P = \bigvee\limits_{i=0}^{M-1} P_i $$

    \begin{table}
        \centering
        \begin{tabular}{|c|c|c|}
            \hline
            \diagbox{$D_i$}{$P_i$} & $0$ & $1$\\
            \hline
            $0$ & \cellcolor{green}-- & \cellcolor{yellow}$P \wedge \neg F$ \\
            \hline
            $1$ & \cellcolor{green}-- & \cellcolor{red}$F$ \\
            \hline
        \end{tabular}
    \end{table}

\end{frame}

    \section{Результаты}
\begin{frame}
    \frametitle{Результаты}
    \begin{itemize}
        \item Спроектирована архитектура системы компьютерного зрения для детектирования опасных ситуаций на производстве;
        \item Разработан метод построения скелетных представлений людей в кадре с использованием модели AlphaPose;
        \item Реализован алгоритм детектирования принадлежности скелетного представления опасной зоне с использованием модели MiDaS;
        \item Описана принцип принятия решений по заданному кадру.
    \end{itemize}
\end{frame}


    \section{Направления улучшения}
\begin{frame}
    \frametitle{Возможные направления дальнейшего улучшения}
    \begin{itemize}
        \item Разработка методов определения иных опасных ситуаций (вроде падения вблизи опасной зоны);
        \item Исследование возможности применения других методов оценки расстояния до объектов в кадре (например, на основе трёхмерной реконструкции помещения);
        \item Реализация сглаживания скелетных представлений на основе межкадрового трекинга для увеличения стабильности работы системы.
    \end{itemize}
\end{frame}

\begin{frame}
    \frametitle{Основная литература}
    \begin{enumerate}
        \item Fang, H. AlphaPose: Whole-Body Regional Multi-Person
        Pose Estimation and Tracking in Real-Time~/ H.\,Fang, J.\,Li, H.\,Tang~// IEEE Transactions on Pattern Analysis and Machine Intelligence.~---
        2023.~--- Vol.\,45, №\,6.~--- P.\,7157--7173.

        \item Ranftl, R. Towards Robust Monocular Depth Estimation: Mixing Datasets for Zero-shot Cross-dataset Transfer~/ R.\,Ranftl, K.\,Lasinger, D.\,Hafner~// IEEE Transactions on Pattern Analysis and Machine Intelligence.~--- 2022.~--- Vol.\,44, №\,3.~--- P.\,1623--1637.

        \item Hartley, R. Multiple View Geometry in Computer Vision~/ R.\,Hartley, A.\,Zisserman.~--- Cambridge University Press, 2004.~--- 670\,p.
    \end{enumerate}

\end{frame}

\end{document}
